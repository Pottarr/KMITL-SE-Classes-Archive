\documentclass[12pt]{report} % You can use 'article' or 'book' class as well

\usepackage{graphicx} % For including images
\usepackage{amsmath}
\usepackage{amssymb}
\usepackage{forest}
\usepackage{multicol}
\begin{document}

% Title page
\begin{titlepage}
	\centering
	\vspace*{1cm} % Adjusts vertical space for the image
	% Insert your image (use the actual path and filename of your image)
	\includegraphics[width=0.3\textwidth]{../images/KMITL Logo.png} % Adjust width as needed

	\vspace{1cm} % Vertical space after the image
	{\LARGE \textbf{Week 4 Homework}} \\[0.5cm] % Title
	\vspace{0.5cm}
	{\large \textbf{Probability Model and Data Analysis}} \\[0.5cm]
    {\large \textbf{Software Engineering Program,}} \\[0.5cm]
	{\large \textbf{Department of Computer Engineering,}} \\[0.5cm]
	{\large \textbf{School of Engineering, KMITL}} \\[1cm]
	{\large 67011352 Theepakorn Phayonrat} \\[0.5cm] % Authors (Use \\ the separate authors)
\end{titlepage}

\section*{Homework of Law of Tatal Probability and Bayes' Theorem (with a tree diagram)}

\subsection*{Context}

The local government of Bangkok wants to interpret ATK test results
from people with symptoms and people who have been in close contact
with people who are known to be infected. It is found that there in
a $70\%$ of a test population having COVID-19 infection. For the ATK
test results, there are not $100\%$ accuracy. The result may be
positive” (a person may test “positive” and not be infected), or the
result may be “false negative” (a person may test “negative” and
however be infected). From the ATKs that are used, it is found that
the probability that a person tests positive if s/he has COVID (a true
positive result) is 0.95, i.e,

$P(Positive \mid COVID) = 0.95$, and

\noindent the probability that a person tests negative if she is not
infected, or healthy (a true negative result) is 0.90, i.e,

$P(Negative \mid healthy) = 0.90$.

\subsection*{Question 1}

Sketch a tree diagram and find the results of the questions $2)$ and
$3)$ using a tree diagram \\

\subsection*{Answer}

From the context given earlier, we know that \\
$\oplus$ is an event when a test person get a positive test result. \\
$\ominus$ is an event when a test person get a negative test result. \\
$C$ is an event when a test person is infected with COVID \\
$H$ is an event when a test person is healthy \\

\newpage

\begin{forest}
for tree={
  grow'=right,                 % Tree grows left to right
  draw, circle,                % All nodes are circles with border
  minimum size=1.2em,          % Consistent node size
  inner sep=1pt,               % Padding inside nodes
  parent anchor=east,          % Connect parent from right side
  child anchor=west,           % Connect child from left side
  edge path={
    \noexpand\path [draw, \forestoption{edge}]
    (!u.parent anchor) -- (.child anchor)\forestoption{edge label};
  },
  s sep=10mm,                  % Sibling separation
  l sep=15mm                   % Level separation
}
% Tree starts here
[, fill=gray!30,
    [$C$, edge label={node[midway,above] {0.70}}
        [$\oplus$, edge label={node[midway,above] {0.95}}
            [$\alpha$]
        ]
        [$\ominus$, edge label={node[midway,below] {0.05}}
            [$\beta$]
        ]
    ]
    [$H$, edge label={node[midway,below] {0.30}}
        [$\oplus$, edge label={node[midway,above] {0.10}}
            [$\gamma$]
        ]
        [$\ominus$, edge label={node[midway,below] {0.90}}
            [$\delta$]
        ]
    ]
]
\end{forest}

\begin{multicols}{2} \notag

\begin{align}
    \therefore $P[C\oplus] & = \alpha \\
    & = (0.70)(0.95) \\
    & = 0.665 \\
    \\
    \therefore $P[H\oplus] & = \gamma \\
    & = (0.30)(0.10) \\
    & = 0.030 \\
    \\
\end{align}

\columnbreak

\begin{align}
    \therefore $P[C\ominus] & = \beta \\
    & = (0.70)(0.05) \\
    & = 0.035 \\
    \\
    \therefore $P[H\ominus] & = \delta \\
    & = (0.30)(0.90) \\
    & = 0.270 \\
    \\
\end{align}

\end{multicols}

\newpage

\subsection*{Question 2}

Find the Probability that a test the person is healthy if s/he
receives a negative test result. Give the answer in a porpotion in
a porpotion in $\%$ by rounding the 2 decimal places.

\subsection*{Solution}

\noindent And from the context given earlier, we also know that \\

\begin{equation} \notag
\begin{split}
    \therefore $P[H \mid \ominus] & = \frac{P[\ominus H]}{P[\ominus]}$ \\
    & = \frac{P[H \ominus]}{P[H \ominus] + P[C \ominus]} \\
    & = \frac{0.270}{0.270 + 0.035} \\
    & = \frac{0.270}{0.305} \\
    & \approx 0.8852 \\
    & = 88.52\% \\
    \\
\end{split}
\end{equation}

\subsection*{Answer}

$\therefore$ the Probability that a test person is healthy receives
a negative test result is in porpotion of $88.52\%$

\newpage

\subsection*{Question 3}

Find the Probability that a test person has COVID if s/he receives
a positive test result. Give the answer in a porpotion in \% by
rounding to 2 decimal places.

\subsection*{Solution}

From the context given earlier, we know that \\

\begin{equation} \notag
\begin{split}
    \therefore $P[C \mid \oplus] & = \frac{P[\oplus C]}{P[\oplus]}$ \\
    & = \frac{P[C \oplus]}{P[H \oplus] + P[C \oplus]} \\
    & = \frac{0.665}{0.030 + 0.665} \\
    & = \frac{0.665}{0.695} \\
    & \approx 0.9568 \\
    & = 95.68\% \\
    \\
\end{split}
\end{equation}


\subsection*{Answer}

$\therefore$ the Probability that a test person has COVID receives
a positive test result is in porpotion of $95.68\%$

\end{document}
